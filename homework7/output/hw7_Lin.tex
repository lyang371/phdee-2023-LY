\documentclass{article}
\usepackage[utf8]{inputenc}
\usepackage{hyperref}
\usepackage[letterpaper, portrait, margin=1in]{geometry}
\usepackage{enumitem}
\usepackage{amsmath}
\usepackage{booktabs}
\usepackage{graphicx}
\usepackage{float}

\usepackage{hyperref}
\hypersetup{
colorlinks=true,
    linkcolor=black,
    filecolor=black,      
    urlcolor=blue,
    citecolor=black,
}
\usepackage{natbib}

\usepackage{titlesec}
  
\title{Homework 7}
\author{Lin Yang}
\date{\today}
  
\begin{document}
\maketitle  
\section{Stata}
~\\
1. (a) 
The coefficient estimate on \textit{treatment\_t} is -0.065 and the heteroskedasticity-robust standard error is 0.0014. 
\\
(b) The coefficient estimate on \textit{treatment\_t} is -0.070 and the heteroskedasticity-robust standard error is 0.0010. 
\\
(c) The possible issue is that although we consider characteristics of zones, day of week, month, hour of day, it may not give us a good match group since not consider year. 


~\\
2. (a) While adding an indicator for year of sample, the coefficient estimate on \textit{treatment\_t} changes to 0.025 and the heteroskedasticity-robust standard error becomes 0.0027.

\\
(b) This attempt addresses the shortcoming of not considering longer term variation such as yearly level and use year variable to match and get a better comparison group. 

~\\
3. (a) By using the nearest neighbour matching, the coefficient of estimating only on the 2020 data is 0.002 and the standard error is 0.0017. 

~\\
(b) The coefficient estimates from regressing the difference in the outcome variable on \textit{treatment\_t} might be strongly influenced by the standard error, so we might choose reasonable number of nearest neighbours rather than just set the number as 1. 

\end{document}
