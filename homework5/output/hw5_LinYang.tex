\documentclass{article}
\usepackage[utf8]{inputenc}
\usepackage{hyperref}
\usepackage[letterpaper, portrait, margin=1in]{geometry}
\usepackage{enumitem}
\usepackage{amsmath}
\usepackage{booktabs}
\usepackage{graphicx}
\usepackage{float}

\usepackage{hyperref}
\hypersetup{
colorlinks=true,
    linkcolor=black,
    filecolor=black,      
    urlcolor=blue,
    citecolor=black,
}
\usepackage{natbib}

\usepackage{titlesec}
  
\title{Homework 5}
\author{Lin Yang}
\date{\today}
  
\begin{document}
\maketitle  
\section{Python}
~\\
1. From the regression outputs, the coefficient of mpg is -22.21, which means that 1 additional unit higher of mpg brings the expected decrease of car price about 22.21 dollars. 

~\\
2. The endogeneity issue may come from omitted variables in the error term, which  is correlated with mpg and also affect the price. For example, the type of car brands are correlated with mpg and also affect price. 

~\\
3. 
(a)The estimated second-stage coefficients, standard errors and first-stage F-statistic of excluded instrument variable: weight are shown in the column 1 of 3(e). \\

(b)The estimated second-stage coefficients, standard errors and first-stage F-statistic of excluded instrument variable: squared weight are shown in the column 2 of 3(e). \\


(c)The estimated second-stage coefficients, standard errors and first-stage F-statistic of excluded instrument variable: height are shown in the column 3 of 3(e).\\

(d)The coefficient of mpg while using weight as instrument variable in (a) and squared weight (b) are similar since weight explain very well for mpg. However, f-statistic for using the height to explain mpg is 0, which means height is not a good instrument variable for exploring the effect of mpg on price. \\

(e)The coefficient of mpg in column 1 and column 2 are similar and significant, which says that weight is good instrument variable. The coefficient of mpg in column3 differed a lot and had a large discrepancy since height might not correlated with mpg a lot. 

\begin{table}[H]
\centering
\begin{tabular}{llll}
\toprule
{} & Weight as IV & Square of Weight as IV & Height as IV \\
\midrule
mpg                 &       150.43 &                 157.06 &     10165.74 \\
                    &      (62.16) &                (62.02) &   (26559.83) \\
car                 &     -4676.09 &               -4732.67 &    -90156.39 \\
                    &     (574.37) &               (573.29) &  (226687.35) \\
constant            &     17627.64 &               17441.23 &   -264024.20 \\
                    &    (1754.87) &              (1751.12) &  (746919.27) \\
First-stage F-stats &        75.46 &                  75.77 &            0 \\
\bottomrule
\end{tabular}

\end{table}

~\\
4. The second-stage coefficient of mpg is 150.43 and the standard error is 63.05 by using GMM with weight as the excluded instrument. 

\section{Stata}
~\\
1. The results table is shown below. 
\begin{table}[H]
\centering
\begin{tabular}{lc} \hline
 & (1) \\
VARIABLES & Impact of Mpg on Price by using predicted length from RDD \\ \hline
 &  \\
rdplot\_hat\_y & 135.41** \\
 & (22.98) \\
car & -3,693.27** \\
 & (225.09) \\
Constant & 17,622.91** \\
 & (739.75) \\
 &  \\
Observations & 1,000 \\
 R-squared & 0.22 \\ \hline
\multicolumn{2}{c}{ Robust standard errors in parentheses} \\
\multicolumn{2}{c}{ ** p$<$0.01, * p$<$0.05} \\
\end{tabular}

\end{table}

~\\
2. From Montiel-Olea-Pflueger IV test, the F-statistic is 78.362 and the 5\% critical value  is 37.41, which means we can reject the null hypothesis of weak instrument at 5\% significance level. Hence, using weight as IV for the effect of mpg on price is reasonable. 


\end{document}
